\documentclass{article}
\usepackage{makeidx}
\usepackage{graphicx}
\makeindex
\title{Referenz Content Management und Bestellsystem}
\author{sven@devcon.net}
\frenchspacing

\begin{document}
\maketitle
\tableofcontents

\pagebreak

\section {Einleitung}

    Dieser Teil der Dokumentation beschreibt recht knapp das System der
    Datenverwaltung und die darauf aufbauende Seitengenerierung zusammen
    mit dem HTML-Spracherweiterungen. Die Benutzerschnittstellen sind
    nicht beschrieben.
    Interessant ist er vor allem f\"ur fortgeschrittene Anwender
    und Entwicklern des Shops.
    Einsteiger seien auf die Schritt-f\"ur-Schritt-Einleitung von Christine
    Sch\"onfeld verwiesen, welche auch die Bedienung erkl\"art.
    
    Haben Sie bereits erste Erfahrungen gesammelt sollten Sie ein wenig
    weiterschm\"okern und sich z.B. Tags ansehen, die Sie bisher noch nicht
    benutzt haben.

    M\"ochten Sie Zusatzmodule f\"ur den Shop entwickeln, so finden Sie hier
    die Grundlagen zum Verst\"andnis der API-Referenz der Bibliotheken,
    welche Sie ben\"otigen, um sich mit den Quelltexten auseinanderzusetzen
    \footnote {Wozu allerdings Programmiererfahrung in PHP oder \"ahnlichen
    Sprachen wie Perl sehr von Nutzen ist.}.


\section {Datenbankverwaltung}

    S\"amtliche Daten die der Shop fasst werden in einer Datenbank abgeleget.
    Damit zwischen Datenbanken verschiedener Hersteller gew\"ahlt werden kann
    benutzt der Shop nur die einfachsten SQL-Befehle und verwaltet die
    Struktur des Daten selbst.
    Es gibt drei grundlegende Datenstrukturen, die der Shop verwaltet:

    \begin{figure}[!h]
    \begin{center}
    \includegraphics[width=\textwidth]{de.ref.db.overview.eps}
    \end{center}
    \end{figure}

    Dazu z\"ahlt das \emph {Verzeichnissystem} und die \emph {Objektverwaltung}
    \index{Content Management System}
    welche ein eigenst\"andiges \emph {Content Management System} (CMS)
    \index{Data Management System}
    \footnote{Vor einiger Zeit von vielen noch als Data-Management-System bezeichnet.}
    bilden und dem \emph {Warenkorb- und Bestellsystem} welches das ganze zu
    einem \emph {Shop}
    \footnote{Zur Verwendung als vollwertiges CMS fehlen noch einige wichtige
    Dinge, so da\ss{} wir beschlossen haben uns nicht allzuweit aus dem
    Fenster zu lehnen.}
    werden l\"asst.

\subsection {Verzeichnisdienst}

    Der Verzeichnisdienst ist aufgebaut wie man es von einem modernen
    \footnote{,,modern'' in dem Sinne, da\ss{} es Verzeichnisse kennt.}
    Dateisystem her kennt. Er verleiht dem gesamten Datenbankinhalt
    \footnote{Schimpfen wir ihn \emph {Content}.} eine
    eindeutige Struktur. Verzeichnisse k\"onnen beliebig viele
    Unterverzeichnisse beinhalten.  Jedes Verzeichnis tr\"agt
    einen Namen.
    Im Gegensatz zum einem \"ublichen Dateisystem gibt es verschiedene
    ,,Verzeichnistypen'', welche bestimmen, was in einem Verzeichnis
    \index{Verzeichnistypen}
    abgelegt werden darf, ob Unterverzeichnisse erlaubt sind, und welche
    Typen von Unterverzeichnissen es geben darf. Die Verzeichnistypen sind
    vordefiniert und k\"onnen nicht vom Benutzer beeinflusst werden
    \footnote{Ausser Sie besitzen den Wagemut eine eigene Version des
    Programms zu entwickeln. Die M\"oglichkeit der Einrichtung eigener
    Verzeichnistypen ist tr\"ostlicherweise geplant.}.
    Verzeichnistypen spielen erst im Zusammenhang mit der dynamischen
    Seitengenerierung eine Rolle, auf welche wir sp\"ater detailliert
    eingehen werden.

\subsection {Objektverwaltung}

    Das Verzeichnissystem kennt keine Dateien. Stattdessen k\"onnen
    sog. \emph {Objekte} in Verzeichnissen abgelegt werden.
    Im Gegensatz zu normalen Dateien erscheint ein Objekt in allen
    Unterverzeichnissen des Verzeichnisses in dem es abgelegt wurde.
    Dieses Verhalten wird als \emph {Vererbung} bezeichnet und
    l\"asst sich f\"ur einzelne Objekte unterdr\"ucken.
    Geerbte Objekte eines Verzeichnisses werden durch das ablegen eines 
    Objekts gleichen Namens ,,\"uberschrieben''.

\subsection {Objektklassen}

    Auch Objekte besitzen, wie Dateien, Namen.
    Diese Objektnamen d\"urfen jedoch nicht frei gew\"ahlt werden, sondern
    m\"ussen registriert sein. Objekt-Namen werden als \emph {Objektklassen}
    bezeichnet.
    Einige Klassen sind bereits zum Zweck der dynamischen
    Seitengenerierung reserviert.

    \begin{figure}[!h]
    \begin{center}
    \begin{tabular}{|l|l|}
	\hline
	l\_index	& Dokument Eingangsseite/Wurzelkategorie \\
	ll\_category	& Aufz\"ahlung Kategorien einer Kategorie \\
	l\_category	& Dokument Kategorie \\
	ll\_page	& Aufz\"ahlung Produktgruppen einer Kategorie \\
	l\_page		& Dokument Produktgruppe \\
	ll\_product	& Aufz\"ahlung Produkte einer Produktgruppe \\
	l\_product	& Dokument Produkt \\
	l\_cart		& Dokument Warenkorb \\
	l\_empty\_cart	& Dokument leerer Warenkorb\\
	ll\_cart	& Aufz\"ahlung bestellter Produkte \\
	l\_order	& Dokument Addressformular/Bestellungsannahme \\
	l\_order\_email	& Dokument Email-Bestellung \\
	l\_order\_confirm & Dokument Email-Best\"atigung f\"ur Kunden \\
	\hline
    \end{tabular}
    \linebreak
    Reservierte Klassen f\"ur die dynamische Seitengenerierung.
    \end{center}
    \end{figure}
    Weitere Klassen dienen der Konfiguration des Bestellsystems.
    \begin{figure}[!h]
    \begin{center}
    \begin{tabular}{|l|l|}
	\hline
	d\_order\_address & Email-Addresse der Bestellungsannahme \\
	d\_order\_duty	& Pflichtfelder im Addressformular \\
	d\_order\_extra & Benutzerdefinierte Felder im Addressformular \\
	\hline
    \end{tabular}
    \linebreak
    Objektklassen zur Konfiguration des Warenkorbes
    \end{center}
    \end{figure}

\pagebreak
\section {Templates}

\subsection {Verzeichnisse und URL-Generierung}

    Wird eine Seite \"uber den Web-Server angefordert, so muss aus der URL
    eindeutig hervorgehen, welche Information dargestellt werden soll.
    Dazu wird der Pfad zum Quellverzeichnis an die URL des Skripts
    angeh\"angt.
    Je nach Typ des angeforderten Verzeichnisses, die sich aus der
    Positionsangabe ergibt, wird eine damit assoziierte
    HTML-Vorlage, ein sogenanntes \emph {Template}, verwendet, in dem
    Platzhalter, sog. Tags, durch Einzelinformationen der Quelle ersetzt
    werden.
    Anschliessend werden Vorlagen an den Client uebermittelt.
    Die Erstellung von URLs kann komplett dem Shop \"uberlassen werden, indem
    entsprechende Tags in den Templates verwendet werden.

    \begin{figure}[!h]
    \begin{center}
    \includegraphics[width=\textwidth]{de.ref.templates.fig.directories.eps}
    Beispielverzeichnis mit verschiedenen Verzeichnistypen und
    Aufz\"ahlungen.
    \end{center}
    \end{figure}
    \index{Verzeichnis}

    Hier eine beispielhafte URL:

    \begin{verbatim}
	http://www.meinshop.de/mod_shop.php3/Futter/Junior_Food/light/
    \end{verbatim}
    Wurde kein Verzeichnis angegeben, so wird das Wurzelverzeichnis zur
    Darstellung der Eingangsseite verwendet:
    \begin{verbatim}
	http://www.meinshop.de/mod_shop.php3
    \end{verbatim}

\subsection {Verzeichnistypen des Shopsystems}

    Alle Verzeichnisse tragen einen Namen 'name'.
    Im Falle vom Typ \emph {PRODUCT} gesellt sich eine Bestellnummer 'key'
    hinzu.

    Die Typen \emph {CATEGORY} und \emph {PRODUCT} d\"urfen Unterverzeichnisse
    gleichen Typs besitzen. Auf jeden Fall besitzen Verzeichnisse vom Typ
    \emph {CATEGORY} Unterverzeichnisse vom Type \emph {PAGE}.
    \emph {PAGE}-Verzeichnisse wiederum besitzen
    \emph {PRODUCT}-Unterverzeichnisse. Alle Verzeichnisse besitzen ein
    virtuelles Verzeichnis \emph {CART} und \emph {ORDER}, welches jeweils eine
    Teilmenge des Mutterverzeichnisses bildet. (Nur Geduld ;)
    \emph {ROOT} stellt das Wurzelverzeichnis der \emph {CATEGORY}-Typen dar.
    Das Wurzelverzeichnis wird zur Darstellung der Startseite benutzt.

\subsection {Vererbbare Dateien - Objekte}

    Mit jedem Verzeichnistyp werden, wie bereits zu Beginn festgestellt,
    Layoutvorlagen assoziiert. Layoutvorlagen werden gerne als
    \emph {Templates}
    (dt. Schablonen) bezeichnet.

\subsection {Dokumente, Teildokumente und Aufz\"ahlungen}

    Templates sind unterteilt in Dokumente und Teildokumente. F\"ur
    ,,Aufz\"ahlungen'' von Unterverzeichnissen, deren Reihenfolge der Benutzer
    manuell festlegen kann, werden stets Teildokumente verwendet. 
    F\"ur jeden Verzeichnistyp gibt es eine Objekt-Klasse f\"ur Dokumente,
    ganze HTML-Seiten z.B., und Aufz\"ahlungen, die jeweils zur Abfrage
    der Templates benutzt werden.
    Die Unterteilung der Templates ist rein logischer Natur. Generell
    k\"onnen beliebige Objekte auch als Templates verwendet werden; selbst,
    wenn es sich bei dem Objekt um eine Graphik handelt, auch wenn dabei
    Unsinn herauskommt.

    F\"ur jeden Verzeichnistyp existieren Standard-Klassen f\"ur Templates:

    \begin{figure}[!h]
    \begin{center}
    \begin{tabular}{|l|l|l|}
	\hline
	CATEGORY, ROOT 	& l\_category 	& Dokument Kategorie \\
			& ll\_category 	& Aufz\"ahlung Kategorien \\
			& l\_index 	& Dokument Eingangsseite \\
	\hline
	PAGE		& l\_page 	& Dokument Produktgruppe \\
			& ll\_page 	& Aufz\"ahlung Produktgruppen \\
	\hline
	PRODUCT 	& l\_product 	& Dokument Produkte \\
			& ll\_product 	& Aufz\"ahlung Produkte \\
	\hline
	CART	 	& l\_cart 	& Dokument Warenkorb \\
			& ll\_cart 	& Aufz\"ahlung bestellter Produkte \\
	\hline
	ORDER 		& l\_order 	& Dokument Bestellungsannahme \\
			& l\_order\_email & Dokument Email-Bestellung \\
			& ll\_order\_confirm & Dokument Email-Best\"atigung f\"ur Kunden \\
	\hline
    \end{tabular}
    \linebreak
    Verzeichnistypen und damit assoziierte Template-Objektklassen
    \end{center}
    \end{figure}

    Der Benutzer kann eigene Objektklassen einf\"uhren und nutzen.
    Ist ein Objekte als ,,\"offentlich'' gekennzeichnet, so kann der Client
    es per HTTP abrufen, z.B. ein Produktfoto.

\pagebreak
\section {Tags}

    Templates enthalten ,,Tags'' um Verzeichnis- oder Objektinformationen
    einzubinden, oder um Aktionen auszul\"osen.
    Tags werden durch Zeichenfolgen ersetzt, welche je nach Zweck aus
    verschiedenen Quellen und auf verschiedene Weisen gewonnen werden.

\subsection {Zeichenketten}
    Zeichenketten, also folgen beliebieger Buchstaben, Zahlen und Sonderzeichen
    unterliegen besonderen Formatierungsregeln, damit die Tags eindeutig
    erkannt werden k\"onnen.
    Soll die Zeichenkette Leerzeichen die Zeichenkette '!$>$' enthalten, so
    muss sie in einfache oder doppelte Anf\"uhrungszeichen eingefasst
    werden. Anf\"uhrungszeichen in der Zeichenkette muss ein Backslash
    vorangestellt werden, auch, wenn die Zeichenkette selbst nicht in
    Anf\"uhrungszeichen gefasst ist.

\subsection {Syntax}

    \begin{verbatim}
	<!QUELLVERZEICHNIS:FUNKTIONSNAME ARGUMENTE!>

	Name der aktuellen Kategorie: <i><!CATEGORY:FIELD name!></i>
    \end{verbatim}

    Tags beschreiben das Quellverzeichnis der abzurufenden Information, die
    darauf anzuwendende Funktion und Gegebenenfalls weitere Zusatzinformationen,
    ,,Argumente'', z.B. der Name einer Tabellenspalte in der Datenbank.
    Tags werden durch die Zeichenfolge '$<$!' eingeleitet und durch '!$>$'
    beendet.

    \subsubsection {Quellverzeichnisse spezifizieren}
    Nach der Einleitung durch '$<!$' kann die Quelle der Daten angegeben werden.
    Quelle kann ein \"ubergeordneter Verzeichnistyp oder ein ,,virtuelles''
    Unterverzeichnis sein, welches in allen Verzeichnissen enthalten ist.
    Dies unterscheidet die \emph {Quelle} von \emph {Verzeichnistypen};
    durch die Angabe eines Verzeichnistyps als Quelle k\"onnen keine
    bestimmten Verzeichnisse angesprochen werden. Gibt man als Quelle
    einen bestimmten Typen an, so ist damit immer das hierarchisch n\"achste
    Verzeichnis gemeint; dessen Mutterverzeichnis ist nicht erreichbar.
    In der Praxis kommt mensch damit gut zurecht.

    \subsubsection {Aktuelle, benachbarte und \"ubergeordnete Verzeichnisse}
    \index{Verzeichnis!aktuelles}
    Ein Quellverzeichnis wird im folgenden als das
    \emph {aktuelle Verzeichnis} bezeichnet.
    \index{Verzeichnis!\"ubergeordnetes Mutterverzeichnis}
    Das einem Verzeichnis
    \"ubergeordnete Verzeichnis wird \emph {Mutterverzeichnis}
    genannt.
    \index{Verzeichnis!Unterverzeichnis}
    Unterverzeichnisse liessen sich entsprechend als
    \emph {Kindverzeichnis} bezeichnen\footnote {Aus humanit\"aren Gr\"unden
    verzichten wir jedoch darauf und bleiben bei der Bezeichnung
    \emph {Unterverzeichnis.}}.
    \index{Verzeichnis!benachbartes}
    Ausser \"ubergeordneten Verzeichnissen gibt es noch benachbarte
    Verzeichnisse, welche von demselben Mutterverzeichnis abh\"angen, wie
    das aktuelle. Diese sind lediglich per Link als eigenes Dokument
    erreichbar.

    \begin{figure}[!h]
    \begin{center}
    \includegraphics[width=0.5\textwidth]{de.ref.templates.fig.siblings.eps}
    \linebreak
    Eine Aufz\"ahlung mit Schl\"usselw\"ortern f\"ur benachbarte Verzeichnisse.
    \end{center}
    \end{figure}

    \index{Verzeichnis!virtuelles}
    Ein weiteres Bonbon sind \emph {virtuelle} Verzeichnisse. Sie
    existieren lediglich einmal, erscheinen jedoch in allen Verzeichnissen
    als Unterverzeichnis. Der Warenkorb z.B. ist als virtuelles Verzeichnis
    realisiert. \emph {Alias}-Quellen bezeichnen Verzeichnisse an einem
    bestimmten Ort, z.B. das Wurzelverzeichnis (\emph {ROOT}), der mit den
    herk\"ommlichen Quellenangaben nicht erreichbar w\"are.

    \index{Aufz\"ahlungen}
    F\"ur andere Typen von Unterverzeichnissen k\"onnen Aufz\"ahlungen generiert
    werden. Innerhalb der Aufz\"ahlungen gilt als aktuelles Verzeichnis stets
    das Unterverzeichnis. Nur \"uber Aufz\"ahlungen lassen sich entsprechend
    Unterverzeichnisse verwenden.

    \index{Verzeichnistypen!Schl\"usselw\"orter}
    Folgende Schl\"usselw\"orter sind zul\"assig:

    \begin{figure}[!h]
    \begin{center}
    \begin{tabular}{|l|l|}
	\hline
	ROOT		& Wurzelkategorie bzw. Eingangsseite (alias)\\
	PARENT		& \"Ubergeordneter Datensatz (alias)\\
	CATEGORY	& Kategorie \\
	PAGE		& Produktgruppe \\
	PRODUCT		& Produkt \\
	CART		& Warenkorb (virtuell)\\
	ORDER		& Adressformular (virtuell)\\
	SESSION		& Sitzung (virtuell)\\
	\hline
    \end{tabular}
    \linebreak
    Schl\"usselw\"orter der Verzeichnistypen zur Quellenangabe.
    \end{center}
    \end{figure}

    Die Angabe der Quelle sollte in nicht generischen Templates der
    \"Ubersichtlichkeit halber immer angegeben werden.

    \subsubsection {Standard-Templates der Verzeichnistypen}
    Zu jeder Quelle existieren zwei Standardklassen, welche ein
    Aufz\"ahlungs- und Dokument-Template enthalten. Nach Bedarf k\"onnen
    auch andere Klassen verwendet werden.
    Klassennamen von Dokumenten beginnen mit 'l\_', Auz\"ahlungen mit 'll\_'.
    (�42 Objektreferenz)

    \begin{figure}[h]
    \begin{center}
    \begin{tabular}{|l|l|}
    	\hline
	ROOT		& l\_root \\
	CATEGORY	& l\_category \\
	PAGE		& l\_page \\
	PRODUCT		& l\_product \\
	CART		& l\_cart \\
	ORDER		& l\_order \\
    	\hline
    \end{tabular}
    \linebreak
    Dokument-Templateklassen
    \end{center}
    \end{figure}

    \begin{figure}[h]
    \begin{center}
    \begin{tabular}{|l|l|}
    	\hline
	CATEGORY	& ll\_category \\
	PAGE		& ll\_page \\
	PRODUCT		& ll\_product \\
	CART		& ll\_cart \\
    	\hline
    \end{tabular}
    \linebreak
    Aufz\"ahlungs-Templateklassen
    \end{center}
    \end{figure}

    Anschliessend muss der Name der auszuf\"uhrenden Informationen
    angegeben werden. Eine \"uebersich finden Sie in der folgenden
    Kurzreferenz. Anschliessend werden die Funktionen detailliert
    beschrieben.

\pagebreak
\subsection {Kurzreferenz Tags}

    \verb$<!:FIELD Feldname!>$

    \verb$<!:OBJECT Klasse!>$

    \verb$<!:TEMPLATE Klasse!>$

    \verb$<!:OBJECTLINK Klasse!>$
    
    \verb$<!:FOUND-OBJECT Klasse!>$
    
    \verb$<!:HAS-OBJECT Klasse!>$
    
    \verb$<!:LINK [Templateklasse]!>$

    \verb$<!:LIST [Templateklasse] [#size[,offset]]!>$

    \verb+<!:FIRST!> Erster Eintrag+

    \verb+<!:PREV!>  Vorheriger Eintrag+

    \verb+<!:NEXT!>  N�chster Eintrag+

    \verb+<!:LAST!>  Letzter Eintrag+

    \verb+<!:PREV-INDEX!> Index des vorherigen Eintrags+

    \verb+<!:INDEX!> Index des aktuellen Eintrags+

    \verb+<!:NEXT-INDEX!> Index des n�chsten Eintrags+

    \verb+<!PRODUCT:KEY!>+

    \verb+<!PRODUCT:QUANTITY!>+

    \verb+<!PRODUCT:FORMQUANTITY!>+

    \verb+<!PRODUCT:QUANTITYNAME!>+

    \verb+<!PRODUCT:PRICE [Waehrung]!>+

    \verb+<!PRODUCT:TOTAL [Waehrung]!>+

    \verb+<!CART:LINK [Menge]!>+

    \verb+<!CART:LIST!>+

    \verb+<!CART:TOTAL [Waehrung]!>+

    \verb+<!CART:QUANTITY [Waehrung]!>+

    \verb+<!CART:IS-EMPTY!>+

    \verb+<!ORDER:SHIPTO-NAME Kurzname!>+

    \verb+<!ORDER:BILLTO-NAME Kurzname!>+

    \verb+<!ORDER:RECEIPTTO-NAME Kurzname!>+

    \verb+<!ORDER:SHIPTO Kurzname!>+

    \verb+<!ORDER:BILLTO Kurzname!>+

    \verb+<!ORDER:RECEIPTTO Kurzname!>+

    \verb+<!SESSION:FORCE!>+

    \verb+<!SESSION:MAGIC-KEYS!>+

\pagebreak
\section {Tag-Funktionen des Content Management Systems}

    Die Tags des Content Management Systems, bestehend aus Objekt-,
    Verzeichnisdienst und Template-Parser k\"onnen mit allen Verzeichnistypen,
    ausgenommen dem Typ $SESSION$, verwendet werden. Manchmal macht die
    Verwendung bestimmter Typen keinen Sinn, was an entsprechender Stelle
    begr\"undet wird.
 
\subsection {Abruf von Datenbankinformationen}

    \subsubsection{Verzeichnisattribute}
    	\verb+<!:FIELD Feldname!>+

	Das folgende Beispiel gibt den Namen einer Kategorie als \"Uberschrift
	aus:
	\begin{verbatim}
    <h3><!CATEGORY:FIELD name</h3>
	\end{verbatim}

    \subsubsection{Objektabruf}
    	\verb+<!:OBJECT Objektklasse!>+

	Das folgende Beispiel bindet ein Objekt benutzerdefinierter Klase
	in das Template ein:
    	\begin{verbatim}
    <!-- HTML-Dokument mit Standard-Kopf und -Fuss -->
    <!:OBJECT html_header!>
      <h3>Hallo Welt!</h3>
        <!-- ...Seiteninhalt... -->
    <!:OBJECT html_footer!>
	\end{verbatim}

    \subsubsection{Templateabruf}
    	\verb+<!:TEMPLATE Objektklasse!>+

	Wie $<$!:OBJECT!$>$, jedoch durchl\"auft das Objekt den Parser im
	Kontext des aufrufenden Templates, falls kein Quellverzeichnistyp
        angegeben wurde. Anschliessend durchl\"auft das Object den
	php-Interpreter.

	\begin{verbatim}
    <!:TEMPLATE u_magic_menu!>
    	\end{verbatim}

    \subsubsection{Aufz\"ahlung untergeordneter Produktinformationen}
	\verb+<!:LIST [Template-Name] [#Anzahl[,Offset]]!>+

	Erzeugt eine Aufz\"ahlung von Unterverzeichnissen des angegebenen
	(Quell-)Typs. Wird keine Template-Klasse angegeben, so wird die
	mit dem Typen assoziierte Standardklasse verwendet.
	Nach dem entsprechenden Template wird von jedem Unterverzeichnis
	aus neu gesucht, so dass sich die Layouts der einzelnen Punkte
	unterscheiden k\"onnen.
	Der Parameter `Anzahl' definiert die Anzahl der aufzulistenden
	Eintr\"age. Der Parameter `Offset' definiert die Nummer des
	ersten anzuzeigenden Eintrags.
	Wird keine Anzahl angegeben, so werden alle Eintr\"age aufgelistet.
	Der Standard-Offset ist der erste Eintrag.

	Hier ein Beispiel, welches die Namen aller Unterkategorien einer
	Kategorie ausgeben soll (Klasse 'l\_category':
	\begin{verbatim}
    <html>
      <head>
        <title>Kategorie <!CATEGORY:FIELD name!></title>
      </head>
      <body>
        <h3>Unterkategorien der Kategorie <!CATEGORY:FIELD name!></h3>
	<!CATEGORY:LIST!>
      </body>
    </html>
 	\end{verbatim}

	Das verwendete Standard-Template $ll\_category$ k\"onnte so aussehen:
	\begin{verbatim}
    <a href="<!CATEGORY:LINK!>"><!CATEGORY:FIELD name!></a><br>
    	\end{verbatim}

\pagebreak
\subsection {Navigation/Links zu Dokumenten und Objekten}

    \subsubsection{Link zu Dokument}
    	\verb+<!:LINK [Template-Name]!>+

	Erzeugt einen Link zu einem \"ubergeordneten Dokument f\"ur
	Produktinformationen. Wird kein Template angegeben, so wird das
	entsprechende Standardtemplate verwendet. (�42)

    \subsubsection{Links zu benachbarten Dokumenten}
    \verb+<!:FIRST!> Erster Eintrag+ \linebreak
    \verb+<!:PREV!>  Vorheriger Eintrag+ \linebreak
    \verb+<!:NEXT!>  N�chster Eintrag+ \linebreak
    \verb+<!:LAST!>  Letzter Eintrag+ \linebreak

    	Diese Tags liefern links zu benachbarten Verzeichnissen, die einem
	gemeinsamen \"ubergeordnetem Eintrag angeh\"oren.

    \subsubsection{$<$!:OBJECTLINK Objektklasse!$>$ Link zu \"offentlichem Objekt}

    	Die Suche nach dem Objekt beginnt ab der angegebenen Quelle.

\pagebreak
\subsection {Strukturinformationen}
    \subsubsection {Indexnummern}

    \verb+<!:PREV-INDEX!> Index des vorherigen Eintrags+ \linebreak
    \verb+<!:INDEX!> Index des aktuellen Eintrags+ \linebreak
    \verb+<!:NEXT-INDEX!> Index des n�chsten Eintrags+ \linebreak
    \verb+<!:LAST-INDEX!> Index des letzten Eintrags/Anzahl der Eintr�ge+ \linebreak

	Alle Eintr\"age, die gemeinsam einem \"ubergeordneten Eintrag
	angeh\"oren, erhalten eine ganzahlige Nummer gr\"osser als 0.
        $<$!:PREV-INDEX!$>$ und $<$!:LAST-INDEX!$>$ werden durch 0 ersetzt,
	falls kein vorheriger oder n\"achster Eintrag existiert.
	\begin{verbatim}
    <b>Indexnummer der vorherigen Seite: <!:PREV-INDEX!></b>
    <b>Indexnummer dieser Seite: <!:INDEX!></b>
    <b>Indexnummer n&auml;chsten Seite: <!:NEXT-INDEX!></b>
    <b>Anzahl der Seiten im aktuellen Mutterverzeichnis: <!:LAST-INDEX!></b>
	\end{verbatim}

    \subsubsection{Aktueller Verzeichnistyp}
    	\verb+<!:TYPE!>+

    	Liefert den Typen des aktuellen Verzeichnisses. M\"ogliche Typen
	sind $CATEGORY$, $PAGE$, $PRODUCT$, $CART$ und $ORDER$.

    \subsubsection{Anzahl direkter Unterverzeichnisse}
    	\verb+<!:NUM-SUBDIRS!>+


	Liefert die Anzahl der Unterverzeichnisse.

	Beispiel:
	\begin{verbatim}
    <b>Anzahl Unterkategorien: <!:NUM-SUBDIRS CATEGORY!>
	\end{verbatim}

    \subsubsection{$<$!:NUM-TYPE Verzeichnistyp!$>$ Anzahl Unterverzeichnesse und Unterverzeichnisse}

	Liefert die Anzahl der Produkte in einer Kategorie und aller
	Unterkategorien oder einer Seite. Verwendet man als Quelle ROOT,
	so wird die Anzahl aller Produkte zur\"uckgegeben.
	Bei fehlender Angabe des Verzeichnistyps wird der aktuelle verwendet.

	Beispiel:
	\begin{verbatim}
    <b>Anzahl Produkte in dieser Kategorie: <!:NUM-TYPE PRODUCT!>
	\end{verbatim}

    \subsubsection{Gefundene Klassen}
    	\verb+<!:FOUND-OBJECT Klasse!>+

	Gibt '1' zur\"uck, wenn ein Objekt der \"ubergebenen Klasse
	gefunden wurde, ansonsten '0'.

	Das folgende Beispiel bindet ein <IMG>-Tag nur dann ein, wenn ein
	das Objekt 'u\_image' existiert:

	\begin{verbatim}
    <? if (<!:FOUND-OBJECT u_image!>) { ?>
    	<img src="<!:OBJECTLINK u_image!>" border="0">
    <? } ?>
    	\end{verbatim}

    \subsubsection{Lokale Klassen}
    	\verb+<!:HAS-OBJECT Klasse!>+

	Gibt '1' zur\"uck, wenn das Quellverzeichnis ein Object der
	angegebenen Klasse besitzt, ansonsten '0'.

    \subsubsection{$<$!:SELECTED Zeichenkette!$>$ Bedingtes Einf\"ugen einer Zeichenkette}

	Liefert die Zeichenkennte ' selected' oder die als Argument \"ubergebene
	Zeichenkette zur\"uck, wenn die Quelle der aktuellen entspricht oder
	letztere der Quelle untergeordnet ist.
	Die Verwendung der Quelle ROOT ist lediglich in generischen Templates
	ohne angabe der Quelle sinnvoll, da alle Eintr\"age der Wurzelkategorie
	untergeordnet sind; die Zeichenkette immer eingef\"ugt wird.
	
	Das folgende Beispiel aktiviert die auszugebende Checkbox, falls die
	wir uns in der entsprechenden Kategorie befinden.

	\begin{verbatim}
    <!-- Aufzaehlung von Kategorien -->
    <option type="checkbox" name="kategoriewahl" value="1" <!:SELECT checked!>>
	\end{verbatim}

\pagebreak
\section {Tag-Funktionen des Bestellsystems}

	Die Tags des Bestellsystems sind immer auf einen bestimmten
	Verzeichnistyp bezogen, weshalb der Quelltyp auch immer angegeben
	ist.

\subsection {Produktmengen}

    \subsubsection{$<$!PRODUCT:QUANTITY!$>$ Gew\"ahlte Menge des Produkts}
    
    	Liefert die Bestellmenge des aktuellen Produkts. Befindet sich das
	Produkt nicht im Warenkorb so wird '0' zur\"uckgegeben.

    \subsubsection{$<$!PRODUCT:QUANTITYNAME!$>$ Name des Mengenfelds zum posten}

    	Name des Formularfeldes des aktuellen Produkts zur \"Ubergabe an
	den Warenkorb. Der Name spezifiert das aktuelle Produkt.

    \begin{verbatim}
    <!-- Dokument Seite mit Produkten -->
    ...
    <form action="<!CART:LINK!>">
      <b>Nennen Sie die gew&uuml;nschte Menge:</b>
        <input type="text"
          name="<!PRODUCT:QUANTITYNAME!>"
          value="<!PRODUCT:QUANTITY!>"
        >
      <p>
        <input type="submit" value="In den Warenkorb legen">
    </form>
    ...
    \end{verbatim}

    \subsubsection{$<$!PRODUCT:FORMQUANTITY!$>$ Menge des bestellten Produkts oder '1'}

	Dieses Tag dient zur \"Ubergabe von Produktmengen an den Korb, ohne
	sie durch eine feste Mengenangabe durcheinanderzubringen.
	Zur\"uckgegeben wird immer '1', falls das Produkt noch nicht
	im Warenkorb liegt. $<$!:QUANTITY!$>$ w\"urde in diesem Fall '0'
	liefern, was zu dem Versuch f\"uhrt ein nicht im Warenkorb abgelegtes
	Produkt zu l\"oschen, was vom Warenkorb wohlwollend ignoriert wird.

    \begin{verbatim}
    <input type="checkbox"
      name="<!PRODUCT:QUANTITYNAME!>"
      value="<!PRODUCT:FORMQUANTITY!>"
    >
    \end{verbatim}

\pagebreak
\subsection {Preise}

    \subsubsection{$<$!PRODUCT:PRICE Waehrung!$>$}

	Liefert den Preis des aktuellen Produkts in der angegebenen
	W\"ahrung. Wird keine W\"ahrung angegeben, so wird die W\"ahrung,
        welche in der Klasse p\_default abgelegt wurde verwendet. Liegt
	kein Preis in der verwendeten W\"ahrung vor, wird ein Weg zur
	Umrechnung gesucht. Ist dies nicht m\"oglich, wird das Tag
	durch den Inhalt des Objekts 'l\_no\_price' zur\"uckgeliefert.

    \subsubsection{$<$!PRODUCT:TOTAL Waehrung!$>$}

	Liefert $<$!PRODUCT:PRICE W\"ahrung!$>$ multipliziert mit
	$<$!PRODUCT;QUANTITY!$>$.
	
    \subsubsection{$<$!CART:TOTAL Waehrung!$>$}

	Summiert den Preis s\"amtlicher bestellten Produkte in der
	angegebenen W\"ahrnug.

\pagebreak
\subsection {Der Warenkorb}

    CART ist ein virtuelles Verzeichnis, welches in allen anderen
    Verzeichnissen enthalten ist.

    \subsubsection{Leerer Warenkorb}

    \verb+<!CART:IS-EMPTY!>+

    Dieses Tag wird durch `1' ersetzt, wenn der Warenkorb keine Produkte
    enth\"aelt, ansonsten `0'. Das folgende Beispiel erzeugt einen Link zum
    Warenkorb, falls dieser nicht leer ist.

    \begin{verbatim}
    	<? if (!<!CART:IS-EMPTY!>) { ?>
		<a href="<!CART!>">Zum Warenkorb</a>
	<? } ?>
    \end{verbatim}

    \subsubsection{$<$!CART:LINK [Produktmenge]!$>$ Link zur Aenderung der Produktmenge}

        Erzeugt einen Link zur \"ubergabe von mehreren Produkten in den
	Warenkorb, falls keine Menge angegeben wurde.
	Hier ein Beispiel eines Warenkorb-Dokuments:
	\begin{verbatim}
    <html>
      <head>
        <title>Warenkorb</title>
      </head>
      <body>
        <h3>Warenkorb</h3>
        <form action="<!CART:LINK!>" method="post">
          <!-- Liste bestellter Produkte -->
	  <!CART:LIST!>
        </form>
        <form action="<!ORDER:LINK!>" method="post">
          <input type="submit" value="bestellen">
        </form>
      </body>
    </html>
	\end{verbatim}
	Als Aufz\"ahlungstemplate kann das Beispiel des Warenkorbs
	\"ubernommen werden.

	Wurde eine Produktmenge \"uebergeben, so wird eine URL erzeugt,
	deren Aufruf die Menge des aktuellen Produkts im Warenkorb \"andert.
	Die Menge kann absolut sein, z.B. 1, um das Produkt in den Warenkorb auf
	zunehmen.
	Eine vorzeichenbehaftete Anzahl, positiv oder negativ, z.B.
	'-1' oder '+2', \"andert die Anzahl relativ zur aktuellen.
	Negative Produktmengen sind nicht m\"oglich.


    \subsubsection{$<$!CART:LIST!$>$ Aufz\"ahlung der Produkte im Warenkorb}

	Aufz\"ahlungen werden ausser im Warenkorb-Dokument auch in der
	Bestellungs- und Best\"atigungsmail verwendet.

\pagebreak
\subsection {Addressinformationen}

    Verzeichnisse vom Typ ORDER sind virtuell.

    \subsubsection{$<$!ORDER:LINK!$>$ Link zum Bestellformular}

    	Liefert die URL des Dokuments, welches die Addressdaten aufnehmen
	soll. \"Ubergebene ECML-Informationen werden gespeichert und k\"onnen
	in das Dokument eingebunden werden.
	Pflichtfelder werden in Objekten der Klasse 'd\_order\_duty'
	definiert.
    
    	Wurde die Bereitschaft zur Bestellungsaufgabe signalisiert
	(siehe $<$!ORDER:FINISH!$>$) und das die Pflichtfelder sind vollst\"andig,
	so wird eine Bestellungsmail an die Email-Adresse im Objekt der
	Klasse 'xyz' sowie eine Best\"atigungsmail an den Kunden versendet und
	das Best\"atigungsdokument ausgegeben. Der Sitzungsschl\"ussel wird
	gesperrt und nicht nochmals verwendet.

	Beispiel eines Addressformulars, welches direkt zur Bestellannahme
	f\"uhrt:
	\begin{verbatim}
    <html>
      <head><title>Bestellformular</title></head>
      <body>
        <form action="<!ORDER!>" method="post">
          <!-- ECML-Version anzeigen -->
            <!ORDER:ECML-VERSION!>
          <!-- Anzeigen, dass eine Bestellungsannahme erfolgen soll, falls
               alle notwendigen Bestellinformationen bereitliegen -->
            <!ORDER:FINISH!>
          <!-- ...Addressfelder... -->
        </form>
      </body>
    </html>
        \end{verbatim}

    \subsubsection{$<$!ORDER:ECML-VERSION!$>$ Benutzte ECML-Version festlegen}

	Generiert ein verstecktes ECML-Formfeld in HTML-Syntax, welches ein
	Formular als ECML v1.1-konform deklariert.

    \subsubsection{$<$!ORDER:FINISH!$>$ Ende der Bestellung in ECML-Formularen}

	Generiert ein verstecktes ECML-Formfeld in HTML-Syntax, welches die
	Bereitschaft zur Bestellungsaufgabe signalisiert. Somit kann sich
	das Addressformular \"uber mehrere Seiten erstrecken.
    	Beispielsweise kann man den Kunden auf eine Seite leiten, die
	lediglich eine Best\"atigung der eingegebenen Informationen verlangt.
	Manche behaupten, dies w\"urde die R\"ucklauf von Produkten
	mindern.

    \subsubsection{ECML-Formfeldnamen}
    \verb+<!ORDER:SHIPTO-NAME Kurzname!>+ \linebreak
    \verb+<!ORDER:BILLTO-NAME Kurzname!>+ \linebreak
    \verb+<!ORDER:RECEIPTTO-NAME Kurzname!>+
    	Liefert den Namen eines der ECML-Addressfelder, die sich eh kein
	Schwein merken kann.

  	S\"amtliche ECML-Formnamen beginnen mit 'Ecom\_'. Je nach Adresstyp
  	kommt noch 'ShipTo\_', 'BillTo\_' oder 'ReceiptTo\_' hinzu.
    \begin{tabular}{|l|l|}
        \hline
  	Kurzname	& Formname \\
  	name\_prefix	& Ecom\_*To\_Postal\_Name\_Prefix \\
  	name\_first	& Ecom\_*To\_Postal\_Name\_First \\
  	name\_middle	& Ecom\_*To\_Postal\_Name\_Middle \\
  	name\_last	& Ecom\_*To\_Postal\_Name\_Last \\
  	name\_suffix	& Ecom\_*To\_Postal\_Name\_Suffix \\
  	company		& Ecom\_*To\_Postal\_Company \\
  	street1		& Ecom\_*To\_Postal\_Street\_Line1 \\
   	street2		& Ecom\_*To\_Postal\_Street\_Line2 \\
  	street3		& Ecom\_*To\_Postal\_Street\_Line3 \\
  	city		& Ecom\_*To\_Postal\_City \\
  	state		& Ecom\_*To\_Postal\_StateProv \\
  	postal\_code	& Ecom\_*To\_Postal\_PostalCode \\
  	country\_code	& Ecom\_*To\_Postal\_CountryCode \\
  	phone		& Ecom\_*To\_Telecom\_Phone\_Number \\
  	email		& Ecom\_*To\_Online\_Email \\
        \hline
    \end{tabular}

    	Beispiel eines Addressfeldes im Formular:
    	\begin{verbatim}
    <!-- ... --->
    <b>Vorname:</b>
    <input type="text" name="<!ORDER:SHIPTO-NAME name_first!>"
                      value="<!ORDER:SHIPTO name_first"><br>
    <b>Name:</b>
    <input type="text" name="<!ORDER:SHIPTO-NAME name_last!>"
                      value="<!ORDER:SHIPTO name_last"><br>
    <!-- ...etc... --->
    	\end{verbatim}

    \subsubsection{Abrufen bereits geposteter Bestelldaten}
    \verb+<!ORDER:SHIPTO Kurzname!>+ \linebreak
    \verb+<!ORDER:BILLTO Kurzname!>+ \linebreak
    \verb+<!ORDER:RECEIPTTO Kurzname!>+

	Wurden bereits ECML-Addressinformationen an den Shop gesendet, so
	stehen diese weiterhin in der Datenbank bereit und k\"onnen mit diesem
	Tag in das Addressformular eingebunden werden.

    \subsubsection{$<$!ORDER:IS-INCOMPLETE!$>$ Unvollst\"andigkeit des Adressformulars}

	Liefert '1' zur\"uck falls das Adressformular unvollst\"andig
	ausgef\"ullt abgesendet wurde, ansonsten '0'. Dies vor allem bei
	Verwendung von php-Code praktisch.

    \subsubsection{$<$!ORDER:ERRORS!$>$ Fehlermeldungen f\"ur unausgef\"ullte Pflichtfelder}

	Liefert die Fehlermeldungen f\"ur unausgef\"ullte Felder in der
	Reihenfolge, wie sie im Objekt der Klasse l\_order\_duty definiert
	sind zur\"uck. Ansonsten wird das Tag verworfen.

\pagebreak
\subsection {Sessions}

    Verzeichnisse vom Typ SESSION sind virtuell.

    \subsubsection{$<$!SESSION:KEY!$>$ Sitzungsschl\"ussel}

	Liefert den Sitzungsschl\"ussel zur\"uck, damit manuell erstellte
	Links damit versehen werden k\"onnen. Als Formularfeldname wird
	stets 'SESSION\_KEY' verwendet.
	\paragraph
	Erste Versionen des Shops verwendeten $<$? echo sid () ?$>$. Diese
	M\"oglichkeit wird nicht mehr unterst\"utzt.

    \subsubsection{$<$!SESSION:FORCE!$>$ Erzwingen einer Sitzung}

	Erzwingt den Einsatz eines einmaligen Schl\"ussels in den nachfolgenden
	Links im Dokument. Dies gilt jedoch nur f\"ur vom Shop-System
	generierte Links. Dieses Tag sollte m\"oglichst am Anfang eines
	Dokuments verwendet werden. Zum Beispiel so:
	\begin{verbatim}
<!SESSION:FORCE!><html>
    <head>
     <!-- ...etc... -->
</html>
	\end{verbatim}

    \subsubsection{$<$!SESSION:MAGIC-KEYS!$>$ Statische Links mit Schl\"usseln versehen}

	Versieht vom Benutzer eingetragene Links im Dokument mit einem
	Session-Key. Da dieser Vorgang relativ zeitaufwendig ist, empfiehlt
	sich die Verwendung von $<$!SESSION:KEY!$>$.
	Es ist v\"ollig egal, in welchem Template dieses Tag benutzt wird.
	Wurde das $<$!SESSION:MAGIC-KEYS!$>$-Tag irgendwo gefunden, werden erst
	nach Erstellung einer Seite manuell eingegebene Links konvertiert.
	Gibt es keinen Sitzungsschl\"ussel, geschieht nichts.
	
	So wird aus ...
	\begin{verbatim}
<!SESSION:MAGIC-KEYS!>
  <a href="extra-angebote.html">Extra-Angebote hier</a>
  	\end{verbatim}
	eine Version mit Sitzungsschl\"usseln:
	\begin{verbatim}
  <a href="extra-angebote.html?SESSION-KEY=2a6sd6dgetc">Extra-Angebote hier</a>
  	\end{verbatim}

	$<$!SESSION:MAGIC-KEYS!$>$ wird durch nichts ersetzt\footnote {
		Es wird einfach entfernt}.
	Es ist darauf zu achten, dass in den neuen Seiten der Schl\"uessel in
	die Links eingetragen ist. Handelt es sich beim Zieldokument um ein
	Dokument benutzerdefinierter Klasse, so kann auch dort
	$<$!SESSION:MAGIC-KEYS!$>$ verwendet werden.

\section {Entwicklung}

    Dieser Abschnitt beschreibt die Entwicklungsumgebung des Shop-Systems.

    Damit das Programm mit kleinstm\"oglichem Aufwand und Vorwissen installiert
    werden kann, wird es vollst\"andig in der Datenbank abgelegt.  Der Benutzer
    muss lediglich ein $Wrapper$-Skript und eine Konfigurationsdatei installiern
    und kann Updates per Web-Interface vornehmen.

    In diese Form wird das Programm jedoch nur mit Hilfe eines Shell-Skripts
    gebracht. Entwickelt wird es wie gewohnt in Dateiform.
    Es durchsucht das Programm nach $include$ und $require$-Befehle und
    f\"ugt die angegeben Dateien an entsprechender Stelle ein. Ein solches
    Image wird vom $Wrapper$ per HTTP oder aus einer Datenbank abgerufen und
    ausgef\"uhrt, wenn die entsprechenden Parameter in der $.dbi.conf$
    eingetragen wurden.

    Der Shop selbst in ein Teil der $central services$, einer Sammlung von
    Webprogrammen und Hilfen zur Installation und Verwaltung.
    S\"amtliche Versionen sind unter $ftp://devcon.net/centralservices/archive$
    zu finden. Entpacken die aktuellste Version in ein Verzeichnis.
    Es enth\"alt die Unterverzeichnisse $local$ und $ports$. $ports$
    enth\"alt eine Sammlung von $Makefiles$ mit deren Hilfe Skripte per
    ftp downgeloadet, notfalls kompiliert und installiert werden k\"onnen.
    Programme welche ein einer Version vorliegen k\"onnen, werden in
    $local$, nach Sprache sortiert, installiert. Damit dies im Falle von
    php-Skripten auch funktioniert, muss der $php_include_path$ auf das
    Verzeichnis local/php zeigen. Die Skripte des Programms, welches hier
    behandelt wird liegen im Verzeichnis local/php/mod\_shop.

    Das folgende Diagramm zeigt die \"Abh\"angigkeiten der Programmteil in
    $local/php$ ausgehen von den Teilen in $local/php/mod_shop$.

    \begin{figure}[!h]
    \begin{center}
    \includegraphics[width=\textwidth]{de.ref.src.overview.eps}
    \end{center}
    \"Ubersicht der Abh\"angigkeiten der wichtigsten Quelldateien.
    \end{figure}

    $mod_shop.php3$ bindet die n\"otigen Teile des Shops ein, nimmt
    Verbindung zur Datenbank auf und ruft das CMS dazu auf eine Seite f\"uer
    die \"ubergebene URL zu generieren.
    Die Initialisierung der Verzeichnistypen und Tags erfolgen global in
    den Dateien $cms.php3$ und $cart.php3$. Die in $mod\_shop.php3$ aufgerufene
    Funktion $process_document$ befindet sich in der Datei $url.php3$.
    Diese Funktion verfolgt den per URL \"ubergebenen Pfad und liest die
    Verzeichnisinformationen in den Context-Cache des Parsers auf den sp\"ater
    mit Hilfe der Template-Tags zugegriffen wird. Im Context-Cache befindet
    sich jeweils der letzte gelesene Eintrag jedes Verzeichnistyps.
    \footnote{Als einfaches assoziatives Array in $DBLAYOUT->context[]$
    realisiert.}
    Nachdem das Dokument-Template bestimmt und eingelesen wurde, kann es
    direkt ausgegebenwerden. Andernfalls ruft die
    Funktion den Parser $DBLAYOUT->parse()$ auf, evaliert die zur\"uckgegebene
    Seite falls das erw\"unscht ist und kehrt zur\"uck zu $mod_shop.php$,
    wo das Skript beendet wird.

    $mod_shop.php3$ enth\"alt der 
\subsection {Datenbankzugriff}

    Datenbankverbindungen werden von der Klasse $DBWrapper$ verwaltet.
    \"Uber sie werden alle SQL-Anfragen abgewickelt, die bei m\"oglichst
    vielen Datenbanken verschiedener Hersteller zum gew\"unschten Ziel
    f\"uhren.
    Komplexere Operationen mit Spalten, Zeilen, Tabellen oder gesamten
    Datenbanken sind mit der Klasse $DBI$ m\"oglich, welche von $DBWRapper$
    abgeleitet ist.
    Sie ist z.B. in der Lage, eine bestimmte Zeile zu l\"oschen, sondern
    auch alle von ihr abh\"angigen Zeilen in anderen Tabellen.
    F\"ur komplexe Operationen m\"ussen einzubeziehnde Tabellen mit dazu
    bereitgestellten Methoden der Klasse beschrieben werden.
    Instanzen der Klasse $DBOBJ$ repr\"asentieren einzelne Objekte. Die
    Hierarchien der Tabellen werden Instanzen der Klasse $DBDEPEND$ bei der
    Konstruktion \"ubergeben.
    Der Parser durchsucht ein Template nach Tags und leitet aus ihnen
    Funktionsnamen ab, die aufgerufen und dessen R\"uckgabewert an Stelle
    der Tags gesetzt werden. Die aufgerufenen Funktionen, $Tag-$ und
    $Dokument-Handler$ befinden sich in den Dateien $cart.php3$,
    $cms.php3$ und $order.php3$. $ecml.php3$ enth\"alt Funktionen zur
    Verwaltung von ECML-Informationen. $send_order.php3$ versendet Bestellungen
    per Mail und kann einfach ausgetauscht werden; z.B. zur Fax-Bestellung.
    $updates.php3$ enth\"alt notfalls Code zum Upgrade von der vorherigen
    Version auf die aktuelle. $attic.php3$ enth\"alt Code, welche in
    zuk\"unftigen Versionen des Shops entfernt werden.

    Das Administrationsskript $admin.php3$ enth\"alt eine von
    $local/php/lib/application.class$ abgeleitete Klasse. $application.class$
    generiert Links und ruft damit verbundene $View-Funktionen$ auf, welche
    mit Hilfe der Klasse $admin_panel$ HTML-Code und -Formulare generiert und
    \"ubergebene Formulare auswertet. Links werden mit Hilfe von Einweg-
    Schl\"usseln, $Tokens$ bezeichnet, generiert, um eine doppelte Ausf\"uhrung
    einer Funktion mit denselben Parametern,
    welche bei der Funktion $\$this->link ()$
    \"ubergeben wurden zu verhindern. Auch Formularfelder erhalten einmalige
    Namen.
    
\subsection {Sessions}
 
    Sessions werden vom CMS verwendet, um virtuelle Verzeichnisse bestimmten
    Usern zuzuordnen. Die Klasse $DBSESSION$ verwaltet Schl\"ussel zusammen
    mit BLOBs.

\subsection {Komplexe Datenbankoperationen}

    Komplexe Datenbankoperationen umfassen das erstellen, ver\"andern, kopieren
    und l\"oschen von Spalten, Zeilen, Tabellen oder Datenbanken unter
    Ber\"ucksichtigung von Referenzen.

\subsection {Verzeichnisse und Listen}

    Zur Zeit steht die Klasse $dbtreeview$ zur Darstellung von Verzeichnissen
    in Administrationsinterfaces bereit.

\subsection {Template-Parser}

    Der Funktion $parse ()$ der Klasse $DBLAYOUT$ durchsucht eine Layoutvorlage
    nach Tags und ruft die daf\"ur zust\"andigen $Tag-Handler$ auf um sie
    durch den R\"uckgabewert zu ersetzen. Gegebenfalls durchl\"auft die
    Vorlage die $eval ()$-Funktion um php-Code auszuf\"uhren.
    Welche Tags erlaubt sind wird \"uber die Funktionen $globcmd ()$,
    $tagcmd ()$ und $assoc ()$ eingestellt. $Dokument-Handler$ initialisieren
    die Ausgabe eines Dokuments.

\subsection {URL-Parser}
\subsection {Das Shop-Modul}
\subsection {Userinterface/Form-Interpreter}
\subsection {Praktische Funktionen}

\section {Glossar}

	\subsection{Vererbung}
	\subsection{Verzeichnis}

\section {Index}
\printindex
\end{document}
